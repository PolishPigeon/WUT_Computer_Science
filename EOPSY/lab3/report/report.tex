\documentclass{report}
\usepackage{listings}
\begin{document}

\section{Scheduling conf settings}
\begin{lstlisting}
// # of Process	
numprocess 2 or 5 or 10

// mean deivation
meandev 2000

// standard deviation
standdev 0

// process    # I/O blocking
process 500
process 500
(more if we set numprocess to 5 or 10)

// duration of the simulation in milliseconds
runtime 10000

\end{lstlisting}

\section{Two processes}
\subsection{Summary Results}


\begin{lstlisting}
Scheduling Type: Batch (Nonpreemptive)
Scheduling Name: First-Come First-Served
Simulation Run Time: 4000
Mean: 2000
Standard Deviation: 0
\end{lstlisting}

\begin{center}
\begin{tabular}{| c | c | c | c | c |}
\hline
Process\# &	CPU Time &	IO Blocking & CPU Completed & CPU Blocked \\
\hline
0&		2000 (ms)&	500 (ms)&	2000 (ms)&	3 times \\
\hline
1&		2000 (ms)&	500 (ms)&	2000 (ms)&	3 times \\
\hline
\end{tabular}
\end{center}

\subsection{Summary Processes}
\begin{lstlisting}
Process: 0 registered... (2000 500 0 0)
Process: 0 I/O blocked... (2000 500 500 500)
Process: 1 registered... (2000 500 0 0)
Process: 1 I/O blocked... (2000 500 500 500)
Process: 0 registered... (2000 500 500 500)
Process: 0 I/O blocked... (2000 500 1000 1000)
Process: 1 registered... (2000 500 500 500)
Process: 1 I/O blocked... (2000 500 1000 1000)
Process: 0 registered... (2000 500 1000 1000)
Process: 0 I/O blocked... (2000 500 1500 1500)
Process: 1 registered... (2000 500 1000 1000)
Process: 1 I/O blocked... (2000 500 1500 1500)
Process: 0 registered... (2000 500 1500 1500)
Process: 0 completed... (2000 500 2000 2000)
Process: 1 registered... (2000 500 1500 1500)
Process: 1 completed... (2000 500 2000 2000)
\end{lstlisting}
\subsection{Comments}
Scheduling type was Batch since I did not change it in SchedulingAlgorithm.java
file \\
Scheduling Name was First-Come First-Served since this is what what we use as
described in README for this laboratory \\
Simulation Run time is 4000 ms and NOT 10000 ms since the simulation finished
before it exceeded this max time \\
Mean is 2000 since this is a value I set in conf value according to laboratory
task description, same with standard deviation equal to 0 and CPU Time equal to
2000 ms \\
IO Blocking is equal to 500 ms, this is a value which we specified in
configuration file and since we did not exceeded the runtime parameter it stayed
equal to 500 ms \\
CPU completed is equal to 2000 since this is deviation we set in configuration
settings and the runtime was not exceeded \\
All processes blocked 3 times, analysing summary processes we can see that they
blocked at 500 ms, 1000 ms and 1500 ms and at 2000 seconds they completed 
 
\section{Five processes}
\newpage
\subsection{Summary Results}
\begin{lstlisting}
Scheduling Type: Batch (Nonpreemptive)
Scheduling Name: First-Come First-Served
Simulation Run Time: 10000
Mean: 2000
Standard Deviation: 0
\end{lstlisting}
\begin{center}                        
\begin{tabular}{| c | c | c | c | c |}                                      
\hline                                                                      
Process\# &     CPU Time &      IO Blocking & CPU Completed & CPU Blocked \\
\hline
0&		2000 (ms)&	500 (ms)&	2000 (ms)&	3 times \\
\hline
1&		2000 (ms)&	500 (ms)&	2000 (ms)&	3 times \\
\hline
2&		2000 (ms)&	500 (ms)&	2000 (ms)&	3 times \\
\hline
3&		2000 (ms)&	500 (ms)&	2000 (ms)&	3 times \\
\hline
4&		2000 (ms)&	500 (ms)&	2000 (ms)&	3 times \\
\hline
\end{tabular}
\end{center}
           
\subsection{Summary Processes}
\begin{lstlisting}
Process: 0 registered... (2000 500 0 0)
Process: 0 I/O blocked... (2000 500 500 500)
Process: 1 registered... (2000 500 0 0)
Process: 1 I/O blocked... (2000 500 500 500)
Process: 0 registered... (2000 500 500 500)
Process: 0 I/O blocked... (2000 500 1000 1000)
Process: 1 registered... (2000 500 500 500)
Process: 1 I/O blocked... (2000 500 1000 1000)
Process: 0 registered... (2000 500 1000 1000)
Process: 0 I/O blocked... (2000 500 1500 1500)
Process: 1 registered... (2000 500 1000 1000)
Process: 1 I/O blocked... (2000 500 1500 1500)
Process: 0 registered... (2000 500 1500 1500)
Process: 0 completed... (2000 500 2000 2000)
Process: 1 registered... (2000 500 1500 1500)
Process: 1 completed... (2000 500 2000 2000)
Process: 2 registered... (2000 500 0 0)
Process: 2 I/O blocked... (2000 500 500 500)
Process: 3 registered... (2000 500 0 0)
Process: 3 I/O blocked... (2000 500 500 500)
Process: 2 registered... (2000 500 500 500)
Process: 2 I/O blocked... (2000 500 1000 1000)
Process: 3 registered... (2000 500 500 500)
Process: 3 I/O blocked... (2000 500 1000 1000)
Process: 2 registered... (2000 500 1000 1000)
Process: 2 I/O blocked... (2000 500 1500 1500)
Process: 3 registered... (2000 500 1000 1000)
Process: 3 I/O blocked... (2000 500 1500 1500)
Process: 2 registered... (2000 500 1500 1500)
Process: 2 completed... (2000 500 2000 2000)
Process: 3 registered... (2000 500 1500 1500)
Process: 3 completed... (2000 500 2000 2000)
Process: 4 registered... (2000 500 0 0)
Process: 4 I/O blocked... (2000 500 500 500)
Process: 4 registered... (2000 500 500 500)
Process: 4 I/O blocked... (2000 500 1000 1000)
Process: 4 registered... (2000 500 1000 1000)
Process: 4 I/O blocked... (2000 500 1500 1500)
Process: 4 registered... (2000 500 1500 1500)
\end{lstlisting}    
\subsection{Comments}
Scheduling type was Batch since I did not change it in SchedulingAlgorithm.java
file \\
Scheduling Name was First-Come First-Served since this is what what we use as
described in README for this laboratory \\
Simulation run time is 10000 ms since it run untill the limit I set in conf file
according to task description \\
CPU blocking is set everywhere to 500 ms as in conf file \\
Mean is 2000 since this is a value I set in conf value according to laboratory
task description, same with standard deviation equal to 0 and CPU Time equal to
2000 ms \\
CPU completed is equal to expected 2000 ms, this makes sense since we had run
time equal to 10000 ms and 5 procesess so each of them could take exactly the
amonunt of time we set them to take.


\section{Ten processes}
\subsection{Summary Results}                                                   
\begin{lstlisting}
Scheduling Type: Batch (Nonpreemptive)
Scheduling Name: First-Come First-Served
Simulation Run Time: 10000
Mean: 2000
Standard Deviation: 0
\end{lstlisting}
\begin{center}                                                                 
\begin{tabular}{| c | c | c | c | c |}                                         
\hline                                                                         
Process\# &     CPU Time &      IO Blocking & CPU Completed & CPU Blocked \\
\hline
0		&2000 (ms)	&500 (ms)&	2000 (ms)&	3 times \\
\hline
1		&2000 (ms)	&500 (ms)&	2000 (ms)&	3 times \\
\hline
2		&2000 (ms)	&500 (ms)&	2000 (ms)&	3 times \\
\hline
3		&2000 (ms)	&500 (ms)&	2000 (ms)&	3 times \\
\hline
4		&2000 (ms)	&500 (ms)&	1000 (ms)&	2 times \\ 
\hline
5		&2000 (ms)	&500 (ms)&	1000 (ms)&	1 times \\
\hline
6		&2000 (ms)	&500 (ms)&	0 (ms)&		0 times \\
\hline
7		&2000 (ms)	&500 (ms)&	0 (ms)&		0 times \\
\hline
8		&2000 (ms)	&500 (ms)&	0 (ms)&		0 times \\
\hline
9		&2000 (ms)	&500 (ms)&	0 (ms)&		0 times \\
\hline
\end{tabular}
\end{center}
\subsection{Summary Processes}
\begin{lstlisting}         
Process: 0 registered... (2000 500 0 0)
Process: 0 I/O blocked... (2000 500 500 500)
Process: 1 registered... (2000 500 0 0)
Process: 1 I/O blocked... (2000 500 500 500)
Process: 0 registered... (2000 500 500 500)
Process: 0 I/O blocked... (2000 500 1000 1000)
Process: 1 registered... (2000 500 500 500)
Process: 1 I/O blocked... (2000 500 1000 1000)
Process: 0 registered... (2000 500 1000 1000)
Process: 0 I/O blocked... (2000 500 1500 1500)
Process: 1 registered... (2000 500 1000 1000)
Process: 1 I/O blocked... (2000 500 1500 1500)
Process: 0 registered... (2000 500 1500 1500)
Process: 0 completed... (2000 500 2000 2000)
Process: 1 registered... (2000 500 1500 1500)
Process: 1 completed... (2000 500 2000 2000)
Process: 2 registered... (2000 500 0 0)
Process: 2 I/O blocked... (2000 500 500 500)
Process: 3 registered... (2000 500 0 0)
Process: 3 I/O blocked... (2000 500 500 500)
Process: 2 registered... (2000 500 500 500)
Process: 2 I/O blocked... (2000 500 1000 1000)
Process: 3 registered... (2000 500 500 500)
Process: 3 I/O blocked... (2000 500 1000 1000)
Process: 2 registered... (2000 500 1000 1000)
Process: 2 I/O blocked... (2000 500 1500 1500)
Process: 3 registered... (2000 500 1000 1000)
Process: 3 I/O blocked... (2000 500 1500 1500)
Process: 2 registered... (2000 500 1500 1500)
Process: 2 completed... (2000 500 2000 2000)
Process: 3 registered... (2000 500 1500 1500)
Process: 3 completed... (2000 500 2000 2000)
Process: 4 registered... (2000 500 0 0)
Process: 4 I/O blocked... (2000 500 500 500)
Process: 5 registered... (2000 500 0 0)
Process: 5 I/O blocked... (2000 500 500 500)
Process: 4 registered... (2000 500 500 500)
Process: 4 I/O blocked... (2000 500 1000 1000)
Process: 5 registered... (2000 500 500 500)
\end{lstlisting}     
\subsection{Comments}
Scheduling type was Batch since I did not change it in SchedulingAlgorithm.java
file \\
Scheduling Name was First-Come First-Served since this is what what we use as
described in README for this laboratory \\
Simulation run time is 10000 ms since it run untill the limit I set in conf file
according to task description \\
IO Blocking set to 500 ms as in config file \\
Mean is 2000 since this is a value I set in conf value according to laboratory
task description, same with standard deviation equal to 0 and CPU Time equal to
2000 ms \\
CPU completed this time is equal to 2000 up to 4th process and then is equal to
1000 ms for 5th and 6th and then it is equal to 0 ms, this means that the
simulation exceeded the runtime before it had a chance to run all processes
\end{document}
